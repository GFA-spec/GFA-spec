\documentclass[10pt]{article}
\usepackage[margin=1in]{geometry}
\usepackage{color}
\definecolor{gray}{rgb}{0.7,0.7,0.7}
\usepackage{framed}
\usepackage{longtable}
\usepackage[pdfborder={0 0 0},hyperfootnotes=false]{hyperref}

\makeindex

\begin{document}

\documentclass[10pt]{article}
\usepackage[margin=1in]{geometry}
\usepackage{color}
\definecolor{gray}{rgb}{0.7,0.7,0.7}
\usepackage{longtable}
\usepackage[pdfborder={0 0 0},hyperfootnotes=false]{hyperref}

\makeindex

\begin{document}

\documentclass[10pt]{article}
\usepackage[margin=1in]{geometry}
\usepackage{color}
\definecolor{gray}{rgb}{0.7,0.7,0.7}
\usepackage{longtable}
\usepackage[pdfborder={0 0 0},hyperfootnotes=false]{hyperref}

\makeindex

\begin{document}

\documentclass[10pt]{article}
\usepackage[margin=1in]{geometry}
\usepackage{color}
\definecolor{gray}{rgb}{0.7,0.7,0.7}
\usepackage{longtable}
\usepackage[pdfborder={0 0 0},hyperfootnotes=false]{hyperref}

\makeindex

\begin{document}

\input{GFAv1.ver}
\title{Graphical Fragment Assembly Format Specification}
\author{The GFA Format Specification Working Group}
\date{\headdate}
\maketitle
\begin{quote}\small
The master version of this document can be found at
\url{https://github.com/pmelsted/GFA-spec}.\\
This printing is version~\commitdesc\ from that repository,
last modified on the date shown above.
\end{quote}
\vspace*{1em}


\section{The GFA Format Specification}

\subsection{An example}

\subsection{Terminology}

\subsection{The Header section}

\subsection{The Sequence section}

\subsection{The Link section}

\subsection{Optional fields}

\section{Recommended Practices for the GFA format}

\section{Reusing existing FASTA files}

\end{document}

\title{Graphical Fragment Assembly Format Specification}
\author{The GFA Format Specification Working Group}
\date{\headdate}
\maketitle
\begin{quote}\small
The master version of this document can be found at
\url{https://github.com/pmelsted/GFA-spec}.\\
This printing is version~\commitdesc\ from that repository,
last modified on the date shown above.
\end{quote}
\vspace*{1em}


\section{The GFA Format Specification}

\subsection{An example}

\subsection{Terminology}

\subsection{The Header section}

\subsection{The Sequence section}

\subsection{The Link section}

\subsection{Optional fields}

\section{Recommended Practices for the GFA format}

\section{Reusing existing FASTA files}

\end{document}

\title{Graphical Fragment Assembly Format Specification}
\author{The GFA Format Specification Working Group}
\date{\headdate}
\maketitle
\begin{quote}\small
The master version of this document can be found at
\url{https://github.com/pmelsted/GFA-spec}.\\
This printing is version~\commitdesc\ from that repository,
last modified on the date shown above.
\end{quote}
\vspace*{1em}


\section{The GFA Format Specification}

\subsection{An example}

\subsection{Terminology}

\subsection{The Header section}

\subsection{The Sequence section}

\subsection{The Link section}

\subsection{Optional fields}

\section{Recommended Practices for the GFA format}

\section{Reusing existing FASTA files}

\end{document}

\title{Graphical Fragment Assembly Format Specification}
\author{The GFA Format Specification Working Group}
\date{\headdate}
\maketitle
\begin{quote}\small
The master version of this document can be found at
\url{https://github.com/pmelsted/GFA-spec}.\\
This printing is version~\commitdesc\ from that repository,
last modified on the date shown above.
\end{quote}
\vspace*{1em}


\section{The GFA Format Specification}
The GFA format is a tab-delimited text format for describing a set of sequences
and their overlap. The primary purpose is to represent the result of an assembly
as a sequence graph, rather than a linear sequence. The first field of the line
identifies the type of the line, segment lines start with `{\tt S}', link lines
start with `{\tt L}' etc.

\subsection{An example}

\subsection{Terminology}

\subsection{The Header section}

\subsection{The Segment section}

\subsection{The Link section}

\subsection{Optional fields}

\section{Recommended Practices for the GFA format}

\section{Reusing existing FASTA files}

\end{document}
